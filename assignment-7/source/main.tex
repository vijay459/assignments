\documentclass{article}
\usepackage[utf8]{inputenc}
\usepackage{circuitikz}

\title{DLD Assignment 7}
\author{Bailapudi Vijay Mani Kumar }
\date{January 2021}

\begin{document}

\maketitle

\section{Boolean Equation}
\begin{equation}
    c=\overline{A}\;B\;\overline{C}\;\overline{D}
\end{equation}

\section{Logic Gate Circuit}
\begin{figure}[h]
    \centering
\begin{circuitikz}
\begin{scope}
\ctikzset{tripoles/american nand port/height=1.6};
\draw 
(0,0)node[american nand port, number inputs=4] (nand1) {};
\end{scope}
\draw
(2,0)node[american nand port] (nand2) {}
(-2.5,1.5)node[american nand port] (nandA) {}
(-2.5,0)node[american nand port] (nandC) {}
(-2.5,-1.5)node[american nand port] (nandD) {}
(-5,1.5)node[](A){A}
(-5,0)node[](C){C}
(-5,-1.5)node[](D){D}

(nand1.in 2) node[anchor=east] {B}
(nand2.out) node[anchor=west] {c}

(nand1.out)--(nand2.in 1)
(nand1.out)--(nand2.in 2)
(A)--++(0:0.5)node[]{}--(nandA.in 1)
(A)--++(0:0.5)node[]{}--(nandA.in 2)
(nandA.out)--(nand1.in 1)
(C)--++(0:0.5)node[]{}--(nandC.in 1)
(C)--++(0:0.5)node[]{}--(nandC.in 2)
(nandC.out)--(nand1.in 3)
(D)--++(0:0.5)node[]{}--(nandD.in 1)
(D)--++(0:0.5)node[]{}--(nandD.in 2)
(nandD.out)--(nand1.in 4)
;
\end{circuitikz}
    \caption{Logic Gate for 2.5.1 (c)}
    \label{fig:gate}
\end{figure}

\end{document}
