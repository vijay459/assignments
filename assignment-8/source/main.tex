\documentclass{article}
\usepackage[utf8]{inputenc}
\usepackage{karnaugh-map}

\title{DLD Assignment 8}
\author{Bailapudi Vijay Mani Kumar}
\date{January 2021}

\begin{document}

\maketitle

\section{Boolean Equation}
\begin{equation}
    c=\overline{A}\;B\;\overline{C}\;\overline{D}
\end{equation}

\section{K - Map}
\begin{figure}[h]
    \centering
    \begin{karnaugh-map}[4][4][1][][]
    \minterms{1}
    \autoterms[0]
    \implicant{4}{14}
    \implicant{12}{10}
    \implicant{3}{10}
    \implicantedge{0}{8}{2}{10}
    \draw[color=black, ultra thin] (0, 4) --
    node [pos=0.7, above right, anchor=south west] {$AB$}
    node [pos=0.7, below left, anchor=north east] {$CD$}
    ++(135:1);
    \end{karnaugh-map}
    \caption{K-Map for 2.5.1 (c)}
    \label{fig:kmap}
\end{figure}
From the K-Map above, we can infer that the POS expression is:
\begin{equation}
    c=\overline{D}\;\overline{C}\;\overline{A}\;B
\end{equation}

\end{document}
